\documentclass[12pt,a4paper,oneside]{article}
\usepackage[cp1250]{inputenc}
\usepackage[english, polish]{babel}
\usepackage{polski}
\usepackage{indentfirst}
\usepackage{float}
\usepackage{hyperref}
\title{In�ynieria j�zyka naturalnego \\ \textbf{Tworzenie mapy dokument�w w j�zyku
polskim w oparciu o metody in�ynierii j�zyka naturalnego}} 
\author{Jacek Miiller, 179265 \and Roman Sorokowski, 179121 \and Bartosz
Micha�ek, 179276}
\date{\today}
\begin{document}
\setlength{\topmargin}{0pt}
\setlength{\headsep }{0pt}
\setlength{\headheight}{0pt}
\setlength{\textheight}{900pt}
\setlength{\oddsidemargin}{10pt}
\setlength{\textwidth}{420pt}
\maketitle

\section*{Wyb�r niezb�dnych narz�dzi i komponent�w programistycznych}

Implementacja projektowanego systemu tworzenia mapy dokument�w wykonana b�dzie w
j�zyku Java. Jako �rodowisko programistyczne pos�u�y �rodowisko Eclipse.

Prace wspomo�e system wersjonowania Git. Repozytorium projektu znajduje si� pod
adresem \url{https://github.com/BlackDragon2/ijn}.

Do ekstrakcji cech z przetwarzanych dokument�w wykorzystana zostanie aplikacja
Fextor.


\end{document}
